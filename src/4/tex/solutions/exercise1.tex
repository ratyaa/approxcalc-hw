\def \picdir{pic/}

\subsection*{Упражнение 1}
\subsubsection*{\Rnum{1}. $I_1$}
Воспользуемся предложенной подстановкой:
\begin{align*}
  I_1 &= \int_{0}^{\infty} \frac{\sin{x}}{x}\; \mathrm{d} x = \int_{0}^{\infty} \mathrm{d} t \int_{0}^{\infty} \Im{\left(e^{-tx + ix}\right)}\; \mathrm{d} x,\\
  J(t) &= \int_{0}^{\infty} \Im{\left(e^{-tx + ix}\right)}\; \mathrm{d} x = - \Im \left(\frac{1}{i-t}\right) = \frac{1}{t^2 + 1},\\
  I_1 &= \int_{0}^{\infty} J(t)\; \mathrm{d} t = \int_{0}^{\infty} \frac{1}{t^2 + 1}\; \mathrm{d} t = \left.\arctan{x}\right|_{0}^{\infty} = \frac{\pi}{2},
\end{align*}
ответ:
\begin{equation*}
  \boxed{I_1 = \frac{\pi}{2}}
\end{equation*}
\subsubsection*{\Rnum{2}. $I_2$}
Для удобства положим $\lambda \ge 0$. Продифференцируем интеграл как функцию от $\lambda$ по $\lambda$:
\begin{equation*}
  \frac{\partial^{} }{\partial \lambda^{}} I_2 = \frac{\partial^{} }{\partial \lambda^{}} \int_{0}^{\infty} \frac{\sin^2{\lambda x}}{x^2}\; \mathrm{d} x = \int_{0}^{\infty} \frac{2\sin{\lambda x}\cos{\lambda x}}{x}\; \mathrm{d} x = \int_{0}^{\infty} \frac{\sin{2\lambda x}}{x}\; \mathrm{d} x = \frac{\pi}{2},
\end{equation*}
откуда получаем
\begin{equation*}
  I_2(\lambda) = \frac{\pi \lambda}{2} + C.
\end{equation*}

Из того, что при $\lambda = 0$ подынтегральная функция тождественно равна нулю, и $I_2(0) = C$, следует $C = 0$.\
Так как подынтегральная функция четна относительно $\lambda$, ответ для произвольного $\lambda$:
\begin{equation*}
  \boxed{I_2 = \frac{\pi |\lambda|}{2}}
\end{equation*}