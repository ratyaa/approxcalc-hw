\def \picdir{pic/}

\subsection*{Упражнение 1}
Заменив переменную, обезразмерим интеграл:
\feq{t = bx,\quad \df x = \frac{1}{b}\df t,\quad I(a, b) = \int_0^\infty e^{-ax} \frac{\sin^2{bx}}{x^2}\; \df x = b \int_0^\infty e^{-\frac{a}{b}t} \frac{\sin^2{t}}{t^2}\; \df t.}

\subsubsection*{\Rnum{1}. $a \gg b$}
При $t \sim \sqrt{\frac{b}{a}} \ll 1$ степень при экспоненте $-\frac{a}{b}t \sim -\sqrt{\frac{a}{b}} \ll -1$,\
следовательно, интеграл набирается при малых $t$ и $\sin^2{t}$ можно разложить в ряд:
\feq{I(a,b) = b \int_0^\infty e^{-\frac{a}{b}t} \frac{\sin^2{t}}{t^2}\; \df t \approx b \int_0^\infty e^{-\frac{a}{b}t} \frac{t^2}{t^2}\; \df t = \frac{b^2}{a} \int_0^\infty e^{-\frac{a}{b}t}\; \df \left(\frac{a}{b}t\right) = \frac{b^2}{a},}
ответ:
\feq{\boxed{I(a,b) \approx \frac{b^2}{a}}}

\subsubsection*{\Rnum{2}. $a \ll b$}
При $t \sim 10$, знаменатель подынтегральной функции $\frac{1}{t^2} \sim 0.01 \ll 1$, а следовательно\
подынтегральная функция $f(t) \ll 1$ и интеграл набирается в некой окрестности $0 \le t \le t' \sim 10$.\
При этом степень при экспоненте остается мала: $-\frac{a}{b}t \ll 1$, следовательно
\feq{e^{-\frac{a}{b}t} \approx 1,\ I(a, b) = b \int_0^\infty e^{-\frac{a}{b}t} \frac{\sin^2{t}}{t^2}\; \df t \approx b \int_0^\infty \frac{\sin^2{t}}{t^2}\; \df t.}
Полученный интеграл вычисляется:
\falign{
  \begin{aligned}
    I &= \int_0^\infty \frac{\sin^2{t}}{t^2}\; \df t = \left.\frac{\sin^2{t}}{t}\right|_0^\infty - \int_0^\infty t\; \df \left(\frac{\sin^2{t}}{t^2}\right) = - \int_0^\infty t\; \df \left(\frac{\sin^2{t}}{t^2}\right)\\
      &= - \int_0^\infty t \left(\frac{2\sin{t}\cos{t}}{t^2} - \frac{2\sin^2{t}}{t^3}\right)\; \df t = - \int_0^\infty \frac{\sin{2t}}{2t}\; \df (2t) + 2I,\quad I = \int_0^\infty \frac{\sin{2t}}{2t}\; \df (2t) = \frac{\pi}{2},
  \end{aligned}
}
ответ:
\feq{\boxed{I(a,b) \approx \frac{\pi b}{2}}}