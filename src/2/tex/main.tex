\documentclass[a4paper, 12pt]{article}

\def \srcdir{tex/}
\def \picdir{pic/}

\input{\srcdir properties}
\input{\srcdir macros}

\title{Домашняя работа \textnumero \input{\srcdir index}}
\author{\input{\srcdir author}}
\date{\input{\srcdir date}}

\begin{document}

\maketitle\thispagestyle{fancy}

% \subsection*{Упражнение 1}
% \lipsum[1-2]

% \subsection*{Упражнение 2}
% \lipsum[3-4]

% \subsection*{Упражнение 3}
% \lipsum[5-6]

% \subsection*{Упражнение 4}
% \lipsum[7-8]

\subsection*{Задача 1}
Определим $f(x)$ как подынтегральную функцию, и $h(x)$ как ее знаменатель:
\salign{I(a) = \int\limits_0^\infty \frac{1}{xa^2 + (1-x^2)^2}\; \df x = \int\limits_0^\infty f(x)\; \df x = \int\limits_0^\infty \frac{1}{h(x)}\; \df x,\quad h(x) = xa^2 + (1-x^2)^2.}
\subsubsection*{\Rnum{1}. $a \gg 1$}
При очень малых $x$, т.е. $xa^2 \ll 1$, следует
\salign{x \ll \frac{1}{a^2} \ll 1 \Rightarrow (1-x^2)^2 \approx 1 \Rightarrow h(x) \sim 1 \Rightarrow f(x) \sim 1.}
При $xa \sim 1$:
\salign{xa^2 \sim a \gg 1,\ x \sim \frac{1}{a} \ll 1 \Rightarrow (1-x^2)^2 \approx 1 \Rightarrow h(x) \sim a \gg 1 \Rightarrow f(x) \ll 1,}
следовательно, интеграл набирается в некой окрестности нуля $x \le x' \sim \frac{1}{a} \ll 1$,\
а значит в $h(x)$ можно отбросить члены высших порядков малости и интегрировать от $0$ до $x'$:
\salign{
  I(a) &= \int\limits_0^\infty \frac{1}{xa^2 + (1-x^2)^2}\; \df x \approx \int\limits_0^{x'} \frac{1}{xa^2 + 1}\; \df x = \frac{1}{a^2}\int\limits_0^{x'} \frac{1}{xa^2 + 1}\; \df (xa^2 + 1) =\\
  &= \left.\frac{1}{a^2}\frac{-1}{(xa^2 + 1)^2}\right|_0^{x'} = \frac{1}{a^2} \left(1 - \frac{1}{(x'a^2 + 1)^2}\right),\ x' \sim \frac{1}{a} \Rightarrow \frac{1}{(x'a^2 + 1)^2} \sim \frac{1}{a^2} \implies %
}
ответ:
\salign{\boxed{I(a) \approx \frac{1}{a^2}}}

\subsubsection*{\Rnum{2}. $a \ll 1$}
При $x \ll 1$ выполняется
\salign{xa^2 \ll 1, (1-x^2)^2 \sim 1 \Rightarrow h(x) \sim 1 \Rightarrow f(x) \sim 1,}
при $x \gg 1$:
\salign{(1-x^2)^2 \gg 1 \Rightarrow h(x) \gg 1 \Rightarrow f(x) \ll 1.}
При $x = 1 + \varepsilon,\ \varepsilon \ll 1$:
\salign{h(x) = (1 + \varepsilon)a^2 + (2\varepsilon + \varepsilon^2)^2 \ll 1 \Rightarrow f(x) \gg 1,}
следовательно, интеграл набирается в некой окрестности $1 - \varepsilon' \le x \le 1 + \varepsilon',\ \varepsilon' \ll 1$,\
а значит в $h(x)$ интегрировать по $\varepsilon$ от $-\varepsilon'$ до $\varepsilon'$, отбросив члены высших порядков малости:
\salign{
  \begin{aligned}
    I(a) = \int\limits_0^\infty \frac{1}{xa^2 + (1-x^2)^2}\; \df x &\approx \int\limits_{-\varepsilon'}^{\varepsilon'} \frac{1}{(1 + \varepsilon)a^2 + (2\varepsilon + \varepsilon^2)^2}\; \df \varepsilon \approx\\
                                                           &\approx \frac12 \int\limits_{-\varepsilon'}^{\varepsilon'} \frac{1}{a^2 + (2\varepsilon)^2}\; \df (2\varepsilon) = \left.\frac{1}{2a}\arctan{\frac{2\varepsilon}{a}}\right|_{-\varepsilon'}^{\varepsilon'} = \frac{1}{a}\arctan{\frac{2\varepsilon'}{a}}.
  \end{aligned}
}
При $\hat\varepsilon \sim \sqrt{a} \hookrightarrow \hat\varepsilon \ll 1 \land \frac{2\hat\varepsilon}{a} \sim \frac{1}{\sqrt{a}} \gg 1$,\
следовательно, приближения выше будут справедливы при $\varepsilon' \ge \hat\varepsilon$ и
\salign{\frac{1}{a}\arctan{\frac{2\varepsilon'}{a} \approx \frac{\pi}{2a}},}
ответ:
\salign{\boxed{I(a) \approx \frac{\pi}{2a}}}
\subsection*{Задача 2}
\subsubsection*{\Rnum{1}. $b \gg a$}
Так как $b \gg a$ и $0 \le x \le a$, $x \ll a \implies$
\salign{e^{\frac{x}{b}} \approx 1 + \frac{x}{b},\quad \int\limits_0^a \frac{x^n}{e^{\frac{x}{b}} - 1}\; \df x \approx \int\limits_0^a bx^{n-1}\; \df x = \left.\frac{bx^n}{n}\right|_0^a = \frac{ba^n}{n},}
ответ:
\salign[*]{\boxed{I(n,a,b) \approx \frac{ba^n}{n}}}

\subsubsection*{\Rnum{2}. $n \gg 1,\ nb \ll a$}
Приблизительно найдем точку $\tilde{x}$, в которой подынтегральная функция
\salign{f(x) = \frac{x^n}{e^{\frac{x}{b}} - 1}}
достигает максимума:
\salign{\left.\frac{\df}{\df x}f(x)\right|_{x=\tilde{x}} = \frac{n\tilde{x}^{n-1}}{e^{\frac{\tilde{x}}{b}} - 1} - \frac{\tilde{x}^ne^{\frac{\tilde{x}}{b}}}{\left(e^{\frac{\tilde{x}}{b}} - 1\right)^2} = 0,\quad \tilde{x} = \frac{nbe^{\frac{\tilde{x}}{b}}}{e^{\frac{\tilde{x}}{b}} - 1}.}
Применяя метод итераций при $\tilde{x}_0 = nb$, так как $n \gg 1$:
\salign{\tilde{x}_1 = \frac{nbe^{\frac{\tilde{x}_0}{b}}}{e^{\frac{\tilde{x}_0}{b}} - 1} = \frac{nbe^n}{e^n - 1} \approx nb = \tilde{x}_0,}
следовательно
\salign{\tilde{x} \approx nb.}

При $x' = \frac{1}{2}nb$:
\salign{\frac{f(x')}{f(\tilde{x})} = \left(\frac12\right)^n \frac{e^n - 1}{e^{\frac{n}{2}} - 1} \approx \left(\frac12\right)^ne^{\frac{n}{2}} = \left(\frac{e}{4}\right)^{\frac{n}{2}} \ll 1,}
при $x'' = 2nb$:
\salign{\frac{f(x'')}{f(\tilde{x})} = 2^{n} \frac{e^n - 1}{e^{2n} - 1} \approx 2^{n}e^{-n} = \left(\frac{2}{e}\right)^n \ll 1,}
следовательно интеграл набирается в некоторой окрестности
\salign{U(\tilde{x}):\ \forall x_u \in U(\tilde{x}) \hookrightarrow \frac12 nb < x_u < 2nb \ll a,}
откуда
\salign[*]{\forall x_u \in U(\tilde{x})\hookrightarrow e^{\frac{n}{2}} - 1 < e^{\frac{x_u}{b}} - 1 \approx e^{\frac{x_u}{b}} \implies}
\salign{\int\limits_0^a \frac{x^n}{e^{\frac{x}{b}} - 1}\; \df x \approx \int\limits_0^a \frac{x^n}{e^{\frac{x}{b}}}\; \df x \approx b^{n+1}\int\limits_0^\infty \left(\frac{x}{b}\right)^ne^{-\frac{x}{b}}\; \df \left(\frac{x}{b}\right) = b^{n+1}\Gamma(n+1).}

В условии задачи не указано, является ли $n$ целым числом (а, следовательно, так как $n \gg 1$ --- натуральным),\
поэтому ответ:
\salign[*]{\boxed{I(n,a,b) \approx b^{n+1}\Gamma(n+1),\text{ для } n \in \mathbb{N}\ I(n,a,b) \approx b^{n+1}n!}}
% = \int\limits_0^a h(x)\; \df x,\ h(x) = x^ne^{-\frac{x}{b}}.}
% Найдем неопределенный интеграл $\int h(x)\; \df x$:
% \salign{\int x^ne^{-\frac{x}{b}}\; \df x = -b\int x^n\; \df e^{- \frac{x}{b}} = -bx^ne^{- \frac{x}{b}} + bn\int x^{n-1}e^{-\frac{x}{b}}\; \df x,}
% откуда по индукции получаем
% \salign{\int h(x)\; \df x = -be^{- \frac{x}{b}}\sum\limits_{k = 0}^n \frac{n!}{(n-k)!} b^k x^{n-k} + C = -be^{- \frac{x}{b}}\sum\limits_{k=0}^n \xi_k(x) + C = F(x).}
% Для $x = 0$:
% \salign{F(0) = -be^{- \frac{x}{b}}\xi_n(0) + C = -b^{n+1}n! + C.}
% Так как $nb \ll a$, и $n \gg 1$, $b \ll \frac{a}{n} \ll a \Rightarrow b \ll a,\ n \ll \frac{a}{b}$, для $x = a$:
% \salign{\forall k \in \{1,\dots,n\} \hookrightarrow \xi_k(a) = \xi_{k-1}(a)\frac{(n-k+1)b}{a} < }
% \svg[0.7]{sin}{Sample Text}
% \svg[0.7]{exp}{Sample Text}
% \svg[0.7]{linexp}{Sample Text}

\end{document}