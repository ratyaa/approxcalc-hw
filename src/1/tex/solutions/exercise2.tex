\def \picdir{pic/}

\subsection*{Упражнение 2}
\subsubsection*{\Rnum{1}. $\alpha \gg 1$}
Пусть $\tilde{x}$ --- корень уравнения $\ln{x} = e^{-\alpha x}$, тогда,
\salign[*]{\forall x \hookrightarrow e^{-\alpha x} \geq 0 \Rightarrow \ln{\tilde{x}} > 0 \Rightarrow \tilde{x} > 1 \Rightarrow e^{-\alpha \tilde{x}} < e^{-\alpha} \ll 1,}
следовательно, $0 < \ln{\tilde{x}} \ll 1$, и $\tilde{x}$ можно представить в виде
\salign{\tilde{x} = 1 + \varepsilon, \quad 0 < \varepsilon \ll 1.}
От полученного данной подстановкой уравнения $\ln{(1 + \varepsilon)} = e^{-\alpha(1 + \varepsilon)}$
возьмем экспоненту:\
\salign[*]{1 + \varepsilon = e^{e^{-\alpha(1 + \varepsilon)}},}
и, так как $\xi = e^{-\alpha(1+\varepsilon)} \ll 1$, разложим правую часть по степеням $\xi$:
\salign[*]{1 + \varepsilon = e^\xi \approx 1 + \xi + \frac{1}{2}\xi^2.}
Подставляя $\xi$ и пренебрегая малыми величинами, получаем
\salign[*]{\varepsilon \approx e^{-\alpha(1+\varepsilon)} + \frac{1}{2}e^{-2\alpha(1+\varepsilon)} \approx e^{-\alpha},}
и, подставляя $\varepsilon$ в (4), получаем ответ:
\salign[*]{\boxed{\tilde{x} \approx 1+e^{-\alpha}}}
\subsubsection*{\Rnum{2}. $\alpha \ll 1$}
Пусть $\tilde{x}$ --- корень уравнения $x-1 = e^{-\alpha x}$, тогда,
\salign[*]{\forall x \hookrightarrow e^{-\alpha x} \geq 0 \Rightarrow \tilde{x} > 1 \Rightarrow |-\alpha \tilde{x}| \ll 1,}
следовательно, $0 < 1 - e^{-\alpha \tilde{x}} \ll 1$, и $e^{-\alpha \tilde{x}}$ можно представить в виде
\salign{e^{-\alpha \tilde{x}} = 1 - \varepsilon, \quad 0 < \varepsilon \ll 1,}
откуда $\tilde{x} = \frac{1}{\alpha}\ln{\frac{1}{1 - \varepsilon}}$, и, подстановкой (5) в исходное уравнение,
\salign{\tilde{x} = e^{1 - \varepsilon},}
\salign[*]{\alpha e^{1 - \varepsilon} = \ln{\frac{1}{1 - \varepsilon}}.}
Пренебрегая малой величиной, получаем
\salign[*]{\alpha e^{1-\varepsilon} \approx \alpha e, \quad \alpha e \approx \ln{\frac{1}{1 - \varepsilon}}, \quad \varepsilon = 1 - e^{-e\alpha} \approx e\alpha,}
откуда подстановкой в (6) получаем ответ:
\salign[*]{\boxed{\tilde{x} \approx e(1 - e\alpha)}}