\message{ !name(main.tex)}\documentclass[a4paper, 12pt]{article}

\def \srcdir{tex/}
\def \picdir{pic/}

\input{\srcdir properties}
\input{\srcdir macros}

\title{Домашняя работа \textnumero \input{\srcdir index}}
\author{\input{\srcdir author}}
\date{\input{\srcdir date}}

\begin{document}

\message{ !name(main.tex) !offset(-3) }


\maketitle\thispagestyle{fancy}

\subsection*{Упражнение 1}
\subsubsection*{\Rnum{1}. $\alpha \gg 1$}
Пусть $\tilde{x}$ --- корень уравнения $x-1 = e^{-\alpha x}$, тогда,
\salign[*]{\forall x \hookrightarrow e^{-\alpha x} \geq 0 \Rightarrow \tilde{x} > 1 \Rightarrow e^{-\alpha \tilde{x}} < e^{-\alpha} \ll 1,}
следовательно, $0 < \tilde{x} - 1 \ll 1$, и $\tilde{x}$ можно представить в виде
\salign{\tilde{x} = 1 + \varepsilon, \quad 0 < \varepsilon \ll 1.}
От полученного данной подстановкой уравнения $\varepsilon = e^{-\alpha(1 + \varepsilon)}$ отбросим малый член:
\salign[*]{e^{-\alpha(1 + \varepsilon)} \approx e{-\alpha}, \quad -\alpha \approx \ln{\varepsilon},}
откуда получаем $\varepsilon \approx e^{-\alpha}$, и, подставляя $\varepsilon$ в (1), получаем ответ:
\salign[*]{\boxed{\tilde{x} \approx 1 + e^{-\alpha}}}

\subsubsection*{\Rnum{2}. $\alpha \ll 1$}
Пусть $\tilde{x}$ --- корень уравнения $x-1 = e^{-\alpha x}$, тогда,
\salign[*]{\forall x \hookrightarrow e^{-\alpha x} \geq 0 \Rightarrow \tilde{x} > 1 \Rightarrow |-\alpha \tilde{x}| \ll 1,}
следовательно, $0 < 1 - e^{-\alpha \tilde{x}} \ll 1$, и $e^{-\alpha \tilde{x}}$ можно представить в виде
\salign{e^{-\alpha \tilde{x}} = 1 - \varepsilon, \quad 0 < \varepsilon \ll 1,}
откуда $\tilde{x} = \frac{1}{\alpha}\ln{\frac{1}{1 - \varepsilon}}$, и, подстановкой (2) в исходное уравнение,
\salign{\tilde{x} = 2 - \varepsilon,}
\salign[*]{\alpha(2 - \varepsilon) = \ln{\frac{1}{1 - \varepsilon}}.}
Пренебрегая малой величиной, получаем
\salign[*]{(2 - \varepsilon)\alpha \approx 2\alpha, \quad 2\alpha \approx \ln{\frac{1}{1 - \varepsilon}}, \quad \varepsilon \approx 1 - e^{-2\alpha},}
откуда подстановкой в (3) получаем ответ:
\salign[*]{\boxed{\tilde{x} \approx 2 - e^{-2\alpha}}}

\subsection*{Упражнение 2}
\subsubsection*{\Rnum{1}. $\alpha \gg 1$}
Пусть $\tilde{x}$ --- корень уравнения $\ln{x} = e^{-\alpha x}$, тогда,
\salign[*]{\forall x \hookrightarrow e^{-\alpha x} \geq 0 \Rightarrow \ln{\tilde{x}} > 0 \Rightarrow \tilde{x} > 1 \Rightarrow e^{-\alpha \tilde{x}} < e^{-\alpha} \ll 1,}
следовательно, $0 < \ln{\tilde{x}} \ll 1$, и $\tilde{x}$ можно представить в виде
\salign{\tilde{x} = 1 + \varepsilon, \quad 0 < \varepsilon \ll 1.}
От полученного данной подстановкой уравнения $\ln{(1 + \varepsilon)} = e^{-\alpha(1 + \varepsilon)}$
возьмем экспоненту:\
\salign[*]{1 + \varepsilon = e^{e^{-\alpha(1 + \varepsilon)}},}
и, так как $\xi = e^{-\alpha(1+\varepsilon)} \ll 1$, разложим правую часть по степеням $\xi$:
\salign[*]{1 + \varepsilon = e^\xi \approx 1 + \xi + \frac{1}{2}\xi^2.}
Подставляя $\xi$ и пренебрегая малыми величинами, получаем
\salign[*]{\varepsilon \approx e^{-\alpha(1+\varepsilon)} + \frac{1}{2}e^{-2\alpha(1+\varepsilon)} \approx e^{-\alpha},}
и, подставляя $\varepsilon$ в (4), получаем ответ:
\salign[*]{\boxed{\tilde{x} \approx 1 + e^{-\alpha}}}

\subsubsection*{\Rnum{2}. $\alpha \ll 1$}
Пусть $\tilde{x}$ --- корень уравнения $x-1 = e^{-\alpha x}$, тогда,
\salign[*]{\forall x \hookrightarrow e^{-\alpha x} \geq 0 \Rightarrow \tilde{x} > 1 \Rightarrow |-\alpha \tilde{x}| \ll 1,}
следовательно, $0 < 1 - e^{-\alpha \tilde{x}} \ll 1$, и $e^{-\alpha \tilde{x}}$ можно представить в виде
\salign{e^{-\alpha \tilde{x}} = 1 - \varepsilon, \quad 0 < \varepsilon \ll 1,}
откуда $\tilde{x} = \frac{1}{\alpha}\ln{\frac{1}{1 - \varepsilon}}$, и, подстановкой (5) в исходное уравнение,
\salign{\tilde{x} = e^{1 - \varepsilon},}
\salign[*]{\alpha e^{1 - \varepsilon} = \ln{\frac{1}{1 - \varepsilon}}.}
Пренебрегая малой величиной, получаем
\salign[*]{\alpha e^{1-\varepsilon} \approx \alpha e, \quad \alpha e \approx \ln{\frac{1}{1 - \varepsilon}}, \quad \varepsilon = 1 - e^{-e\alpha},}
откуда подстановкой в (6) получаем ответ:
\salign[*]{\boxed{\tilde{x} \approx e^{e^{-e\alpha}}}}

\subsection*{Упражнение 3}
\subsubsection*{\Rnum{1}. $x_1(\lambda)$}
\subsubsection*{\Rnum{2}. $x_2(\lambda)$}
Домножим обе части на $-1$ и прологарифмируем обе части уравнения:
\salign{x = \ln{(-\lambda)} - \ln{(-x)}.}
Чтобы записи были приятнее глазу, перепишем уравнение в следующем виде:
\salign{y = -x, \quad \xi = -\ln{(-\lambda)}, \quad y = \xi + \ln{y}}.
Пусть $\tilde{y}$ --- корень уравнения, удовлетворяющий условию задачи\
$x_2(\lambda) < -1 \Leftrightarrow \tilde{y} > 1$. Применим метод итераций к (нумбер) и докажем, что\
последовательность ${y_n}$ сходится в $\tilde{y}$. Пусть $y_1 = 1$, тогда
\salign{y_2 = \xi + \ln{y_1}, \quad (\Delta y)_1 = y_2 - y_1 = \xi - 1.}
Так как $|\lambda| \ll 1$, $\xi > 1$ и, следовательно,
\salign{y_2 > y_1, \quad (\Delta y)_1 > 0.}
Предположим, что $\exists n: y_n > \tilde{y}$, тогда
\salign{y_n = \xi + \ln{y_{n-1}}, \quad y_{n-1} = e^{y_n - \xi} = e^{(\tilde{y} - \xi) + (y_n - \tilde{y})} = \tilde{y}e^{y_n - \tilde{y}} > \tilde{y}},
откуда по индукции:
\salign{\exists n: y_n > \tilde{y} \Rightarrow \forall k \in \mathbb{N}, \ k \leq n \hookrightarrow y_k > \tilde{y},}
что противоречит начальным условиям $\tilde{y} > 1, \ y_1 = 1$.

\subsection*{Упражнение 4}
\lipsum[7-8]

\subsection*{Задача 1}
\lipsum[9-10]

\subsection*{Задача 2}
\lipsum[11-12]

\end{document}
\message{ !name(main.tex) !offset(-102) }
