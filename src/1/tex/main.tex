\documentclass[a4paper, 12pt]{article}

\def \srcdir{tex/}
\def \picdir{pic/}

\input{\srcdir properties}
\input{\srcdir macros}

\title{Домашняя работа \textnumero \input{\srcdir index}}
\author{\input{\srcdir author}}
\date{\input{\srcdir date}}

\begin{document}

\maketitle\thispagestyle{fancy}

\subsection*{Упражнение 1}
\subsubsection*{\Rnum{1}. $\alpha \gg 1$}
Пусть $x_0$ --- корень уравнения $x-1 = e^{-\alpha x}$, тогда,
\salign[*]{\forall x \hookrightarrow e^{-\alpha x} \geq 0 \implies x_0 > 1 \implies e^{-\alpha x_0} < e^{-\alpha} \ll 1,}
следовательно, $0 < x_0 - 1 \ll 1$, и $x_0$ можно представить в виде
\salign{x_0 = 1 + \varepsilon, \quad 0 < \varepsilon \ll 1.}
От полученного данной подстановкой уравнения $\varepsilon = e^{-\alpha(1 + \varepsilon)}$ отбросим малый член:
\salign[*]{e^{-\alpha(1 + \varepsilon)} \approx e{-\alpha}, \quad -\alpha \approx \ln{\varepsilon},}
откуда получаем $\varepsilon \approx e^{-\alpha}$, и, подставляя $\varepsilon$ в (1), получаем ответ:
\salign[*]{\boxed{x_0 \approx 1 + e^{-\alpha}}}

\subsubsection*{\Rnum{2}. $\alpha \ll 1$}
Пусть $x_0$ --- корень уравнения $x-1 = e^{-\alpha x}$, тогда,
\salign[*]{\forall x \hookrightarrow e^{-\alpha x} \geq 0 \implies x_0 > 1 \implies |-\alpha x_0| \ll 1,}
следовательно, $0 < 1 - e^{-\alpha x_0} \ll 1$, и $e^{-\alpha x_0}$ можно представить в виде
\salign{e^{-\alpha x_0} = 1 - \varepsilon, \quad 0 < \varepsilon \ll 1,}
откуда $x_0 = \frac{1}{\alpha}\ln{\frac{1}{1 - \varepsilon}}$, и, подстановкой (2) в исходное уравнение,
\salign{x_0 = 2 - \varepsilon,}
\salign[*]{\alpha(2 - \varepsilon) = \ln{\frac{1}{1 - \varepsilon}}.}
Пренебрегая малой величиной, получаем
\salign[*]{(2 - \varepsilon)\alpha \approx 2\alpha, \quad 2\alpha \approx \ln{\frac{1}{1 - \varepsilon}}, \quad \varepsilon \approx 1 - e^{-2\alpha},}
откуда подстановкой в (3) получаем ответ:
\salign[*]{\boxed{x_0 \approx 2 - e^{-2\alpha}}}

\subsection*{Упражнение 2}
\subsubsection*{\Rnum{1}. $\alpha \gg 1$}
Пусть $x_0$ --- корень уравнения $\ln{x} = e^{-\alpha x}$, тогда,
\salign[*]{\forall x \hookrightarrow e^{-\alpha x} \geq 0 \implies \ln{x_0} > 0 \implies x_0 > 1 \implies e^{-\alpha x_0} < e^{-\alpha} \ll 1,}
следовательно, $0 < \ln{x_0} \ll 1$, и $x_0$ можно представить в виде
\salign{x_0 = 1 + \varepsilon, \quad 0 < \varepsilon \ll 1.}
От полученного данной подстановкой уравнения $\ln{(1 + \varepsilon)} = e^{-\alpha(1 + \varepsilon)}$
возьмем экспоненту:\
\salign[*]{1 + \varepsilon = e^{e^{-\alpha(1 + \varepsilon)}},}
и, так как $\xi = e^{-\alpha(1+\varepsilon)} \ll 1$, разложим правую часть по степеням $\xi$:
\salign[*]{1 + \varepsilon = e^\xi \approx 1 + \xi + \frac{1}{2}\xi^2.}
Подставляя $\xi$ и пренебрегая малыми величинами, получаем
\salign[*]{\varepsilon \approx e^{-\alpha(1+\varepsilon)} + \frac{1}{2}e^{-2\alpha(1+\varepsilon)} \approx e^{-\alpha},}
и, подставляя $\varepsilon$ в (4), получаем ответ:
\salign[*]{\boxed{x_0 \approx 1 + e^{-\alpha}}}

\subsubsection*{\Rnum{2}. $\alpha \ll 1$}
Пусть $x_0$ --- корень уравнения $x-1 = e^{-\alpha x}$, тогда,
\salign[*]{\forall x \hookrightarrow e^{-\alpha x} \geq 0 \implies x_0 > 1 \implies |-\alpha x_0| \ll 1,}
следовательно, $0 < 1 - e^{-\alpha x_0} \ll 1$, и $e^{-\alpha x_0}$ можно представить в виде
\salign{e^{-\alpha x_0} = 1 - \varepsilon, \quad 0 < \varepsilon \ll 1,}
откуда $x_0 = \frac{1}{\alpha}\ln{\frac{1}{1 - \varepsilon}}$, и, подстановкой (5) в исходное уравнение,
\salign{x_0 = e^{1 - \varepsilon},}
\salign[*]{\alpha e^{1 - \varepsilon} = \ln{\frac{1}{1 - \varepsilon}}.}
Пренебрегая малой величиной, получаем
\salign[*]{\alpha e^{1-\varepsilon} \approx \alpha e, \quad \alpha e \approx \ln{\frac{1}{1 - \varepsilon}}, \quad \varepsilon = 1 - e^{-e\alpha},}
откуда подстановкой в (6) получаем ответ:
\salign[*]{\boxed{x_0 \approx e^{e^{-e\alpha}}}}

\subsection*{Упражнение 3}
\lipsum[5-6]

\subsection*{Упражнение 4}
\lipsum[7-8]

\subsection*{Задача 1}
\lipsum[9-10]

\subsection*{Задача 2}
\lipsum[11-12]

\svg[0.7]{sin}{Sample Text}
\svg[0.7]{exp}{Sample Text}
\svg[0.7]{linexp}{Sample Text}

\end{document}