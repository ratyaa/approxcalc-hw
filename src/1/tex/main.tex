\documentclass[a4paper, 12pt]{article}

\def \srcdir{tex/}
\def \picdir{pic/}

\input{\srcdir properties}
\input{\srcdir macros}

\title{Домашняя работа \textnumero \input{\srcdir index}}
\author{\input{\srcdir author}}
\date{\input{\srcdir date}}

\begin{document}

\maketitle\thispagestyle{fancy}

\subsection*{Упражнение 1}
\subsubsection*{\Rnum{1}. $\alpha \gg 1$}
Пусть $\tilde{x}$ --- корень уравнения $x-1 = e^{-\alpha x}$, тогда,
\salign[*]{\forall x \hookrightarrow e^{-\alpha x} \geq 0 \Rightarrow \tilde{x} > 1 \Rightarrow e^{-\alpha \tilde{x}} < e^{-\alpha} \ll 1,}
следовательно, $0 < \tilde{x} - 1 \ll 1$, и $\tilde{x}$ можно представить в виде
\salign{\tilde{x} = 1 + \varepsilon, \quad 0 < \varepsilon \ll 1.}
От полученного данной подстановкой уравнения $\varepsilon = e^{-\alpha(1 + \varepsilon)}$ отбросим малый член:
\salign[*]{e^{-\alpha(1 + \varepsilon)} \approx e{-\alpha}, \quad -\alpha \approx \ln{\varepsilon},}
откуда получаем $\varepsilon \approx 1 - \alpha$, и, подставляя $\varepsilon$ в (1), получаем ответ:
\salign[*]{\boxed{\tilde{x} \approx 2 - \alpha}}
\subsubsection*{\Rnum{2}. $\alpha \ll 1$}
Пусть $\tilde{x}$ --- корень уравнения $x-1 = e^{-\alpha x}$, тогда,
\salign[*]{\forall x \hookrightarrow e^{-\alpha x} \geq 0 \Rightarrow \tilde{x} > 1 \Rightarrow |-\alpha \tilde{x}| \ll 1,}
следовательно, $0 < 1 - e^{-\alpha \tilde{x}} \ll 1$, и $e^{-\alpha \tilde{x}}$ можно представить в виде
\salign{e^{-\alpha \tilde{x}} = 1 - \varepsilon, \quad 0 < \varepsilon \ll 1,}
откуда $\tilde{x} = \frac{1}{\alpha}\ln{\frac{1}{1 - \varepsilon}}$, и, подстановкой (2) в исходное уравнение,
\salign{\tilde{x} = 2 - \varepsilon,}
\salign[*]{\alpha(2 - \varepsilon) = \ln{\frac{1}{1 - \varepsilon}}.}
Пренебрегая малой величиной, получаем
\salign[*]{(2 - \varepsilon)\alpha \approx 2\alpha, \quad 2\alpha \approx \ln{\frac{1}{1 - \varepsilon}}, \quad \varepsilon \approx 1 - e^{-2\alpha} \approx 2\alpha,}
откуда подстановкой в (3) получаем ответ:
\salign[*]{\boxed{\tilde{x} \approx 2 - 2\alpha}}
\subsection*{Упражнение 2}
\subsubsection*{\Rnum{1}. $\alpha \gg 1$}
Пусть $\tilde{x}$ --- корень уравнения $\ln{x} = e^{-\alpha x}$, тогда,
\salign[*]{\forall x \hookrightarrow e^{-\alpha x} \geq 0 \Rightarrow \ln{\tilde{x}} > 0 \Rightarrow \tilde{x} > 1 \Rightarrow e^{-\alpha \tilde{x}} < e^{-\alpha} \ll 1,}
следовательно, $0 < \ln{\tilde{x}} \ll 1$, и $\tilde{x}$ можно представить в виде
\salign{\tilde{x} = 1 + \varepsilon, \quad 0 < \varepsilon \ll 1.}
От полученного данной подстановкой уравнения $\ln{(1 + \varepsilon)} = e^{-\alpha(1 + \varepsilon)}$
возьмем экспоненту:\
\salign[*]{1 + \varepsilon = e^{e^{-\alpha(1 + \varepsilon)}},}
и, так как $\xi = e^{-\alpha(1+\varepsilon)} \ll 1$, разложим правую часть по степеням $\xi$:
\salign[*]{1 + \varepsilon = e^\xi \approx 1 + \xi + \frac{1}{2}\xi^2.}
Подставляя $\xi$ и пренебрегая малыми величинами, получаем
\salign[*]{\varepsilon \approx e^{-\alpha(1+\varepsilon)} + \frac{1}{2}e^{-2\alpha(1+\varepsilon)} \approx e^{-\alpha} \approx 1 - \alpha,}
и, подставляя $\varepsilon$ в (4), получаем ответ:
\salign[*]{\boxed{\tilde{x} \approx 2 - \alpha}}
\subsubsection*{\Rnum{2}. $\alpha \ll 1$}
Пусть $\tilde{x}$ --- корень уравнения $x-1 = e^{-\alpha x}$, тогда,
\salign[*]{\forall x \hookrightarrow e^{-\alpha x} \geq 0 \Rightarrow \tilde{x} > 1 \Rightarrow |-\alpha \tilde{x}| \ll 1,}
следовательно, $0 < 1 - e^{-\alpha \tilde{x}} \ll 1$, и $e^{-\alpha \tilde{x}}$ можно представить в виде
\salign{e^{-\alpha \tilde{x}} = 1 - \varepsilon, \quad 0 < \varepsilon \ll 1,}
откуда $\tilde{x} = \frac{1}{\alpha}\ln{\frac{1}{1 - \varepsilon}}$, и, подстановкой (5) в исходное уравнение,
\salign{\tilde{x} = e^{1 - \varepsilon},}
\salign[*]{\alpha e^{1 - \varepsilon} = \ln{\frac{1}{1 - \varepsilon}}.}
Пренебрегая малой величиной, получаем
\salign[*]{\alpha e^{1-\varepsilon} \approx \alpha e, \quad \alpha e \approx \ln{\frac{1}{1 - \varepsilon}}, \quad \varepsilon = 1 - e^{-e\alpha} \approx e\alpha,}
откуда подстановкой в (6) получаем ответ:
\salign[*]{\boxed{\tilde{x} \approx e^{1 - e\alpha}}}
\subsection*{Упражнение 3}
\subsubsection*{\Rnum{1}. $\tilde{x}_1(\lambda)$}
Пусть $0 > \tilde{x} > -1$ --- корень уравнения $xe^x = \lambda$. При малых $x$ таких, что $|x| \ll 1$,\
\salign[*]{x e^x \approx x(1 + x + \frac{1}{2}x^2) \approx x,}
следовательно, $\tilde{x} \approx \tilde{x}e^x \approx \lambda$, ответ:
\salign[*]{\boxed{\tilde{x} \approx \lambda}}
\subsubsection*{\Rnum{2}. $\tilde{x}_2(\lambda)$}
Домножим на $-1$ и прологарифмируем обе части уравнения:
\salign[*]{x = \ln{(-\lambda)} - \ln{(-x)}.}
Чтобы записи были приятнее глазу, перепишем уравнение в следующем виде:
\salign{y = -x, \quad \xi = -\ln{(-\lambda)}, \quad y = \xi + \ln{y}.}

Пусть $\tilde{y}$ --- корень данного уравнения, удовлетворяющий условию задачи\
$x_2(\lambda) < -1 \Leftrightarrow \tilde{y} > 1$. Применим метод итераций к (7) и докажем, что\
последовательность $\{y_n\}$ сходится в $\tilde{y}$.

Пусть $y_1 = 1$, тогда
\salign[*]{y_2 = \xi + \ln{y_1}, \quad (\Delta y)_1 = y_2 - y_1 = \xi - 1.}
Так как $|\lambda| \ll 1$, $\xi > 1$ и, следовательно,
\salign[*]{y_2 > y_1, \quad (\Delta y)_1 > 0.}
Докажем ограниченность сверху для $\{y_n\}$. Предположим, что $\exists n: y_n \geq \tilde{y}$, тогда
\salign[*]{y_n = \xi + \ln{y_{n-1}}, \quad y_{n-1} = e^{y_n - \xi} = e^{(\tilde{y} - \xi) + (y_n - \tilde{y})} = \tilde{y}e^{y_n - \tilde{y}} \geq \tilde{y},}
откуда по индукции:
\salign[*]{\exists n: y_n \geq \tilde{y} \Rightarrow \forall k \in \mathbb{N}, \ k \leq n \hookrightarrow y_k \geq \tilde{y},}
что противоречит начальным условиям $\tilde{y} > 1, \ y_1 = 1$, следовательно $\{y_n\}$\
ограниченна сверху, и
\salign{\forall n \in \mathbb{N} \hookrightarrow y_n < \tilde{y}.}

Докажем, что последовательность $\{y_n\}$ монотонно возрастает:
\salign[*]{(\Delta y)_n = y_{n+1} - y_n = (\xi + \ln{y_n}) - (\xi + \ln{y_{n-1}}) = \ln{\frac{y_n}{y_{n-1}}} = \ln{\left(1 + \frac{(\Delta y)_{n-1}}{y_{n-1}}\right)},}
\salign[*]{(\Delta y)_{n-1} > 0 \Rightarrow \ln\left(1 + \frac{(\Delta y)_{n-1}}{y_{n-1}}\right) > 0 \Rightarrow (\Delta y)_n > 0,}
откуда по индукции:
\salign[*]{(\Delta y)_1 > 0 \Rightarrow \forall n \in \mathbb{N} \hookrightarrow (\Delta y)_n > 0 \Leftrightarrow \forall n \in \mathbb{N} \hookrightarrow y_{n+1} > y_n,}
т.е. $\{y_n\}$ монотонно возрастает.

Oтсюда и из условия (8) ограниченности последовательности $\{y_n\}$
\salign[*]{\exists \lim_{n \to \infty} y_n = \hat{y}, \quad \hat{y} \leq \tilde{y}.}
Переходя к пределу в формуле $y_{n+1} = \xi + \ln{y_n}$ при $n \to \infty$, получаем\
$\hat{y} = \xi + \ln{\hat{y}}$, т.е. $\hat{y} = \tilde{y}$. $\blacksquare$

То, с какой точностью записывать ответ, сильно зависит от $\lambda$. Например, при\
$x_1 = -1$ для $\lambda = -e^{-4}$, с точностью до трёх значащих цифр
\salign[*]{\tilde{x}_2(-e^{-4}) \approx x_6 = -(\xi + \ln{(\xi + \ln{(\xi + \ln{(\xi + \ln{\xi})})})}), \ \xi = -\ln{(-\lambda)} = 4,}
a для $\lambda = -e^{-8}$ ту же точность получаем при
\salign[*]{\tilde{x}_2(-e^{-8}) \approx x_4 = -(\xi + \ln{(\xi + \ln{\xi})}), \ \xi = -\ln{(-\lambda)} = 8.}
\textbf{Буду считать, что $\lambda = -e^{-4} \approx -0.02$ удовлетворяет условию $|\lambda| \ll 1$ и запишу в ответ $\tilde{x}_2(\lambda) \approx x_6$:}
\salign[*]{\boxed{\tilde{x}_2(\lambda) \approx -(\xi + \ln{(\xi + \ln{(\xi + \ln{(\xi + \ln{\xi})})})}), \ \xi = -\ln{(-\lambda)}}}
\subsection*{Упражнение 4}
Пусть $\tilde{x}$ --- корень данного уравнения, а $\lambda = - \frac{1}{e} + \delta, 0 < \delta \ll 1$,\
тогда $\tilde{x}$ можно представить в виде
\salign{\tilde{x} = -1 + \varepsilon, \quad |\varepsilon| \ll 1.}
Домножим обе части уравнения на $e$ и разложим левую часть по степеням $\varepsilon$:
\salign[*]{(-1 + \varepsilon)e^\varepsilon \approx (-1 + \varepsilon)(1 + \varepsilon + \frac{1}{2}\varepsilon^2) \approx -1 + \delta e,}
\salign[*]{\frac{1}{2}\varepsilon^2 = \delta e, \quad \varepsilon = \pm \sqrt{2\delta e} = \pm \sqrt{2(\lambda e + 1)},}
отсюда, подставляя в (9) и учитывая, что $\tilde{x}_1(\lambda) > \tilde{x}_2(\lambda)$, ответ:
\salign[*]{\boxed{\tilde{x}_1(\lambda) = -1 + \sqrt{2(\lambda e + 1)}, \quad \tilde{x}_2(\lambda) = -1 - \sqrt{2(\lambda e + 1)}}}
\subsection*{Задача 1}
\subsubsection*{\Rnum{1}. $0 < \alpha -1 \ll 1$}
Нарисуем на графике правую часть уравнения, и левую при $\alpha \in \{2,\ 1,\ \frac12\}$\
(\textbf{рис. 1}). Вблизи нуля функции $\tanh{\alpha x}$ и $\arctan{x}$ ведут себя линейно:\
$\tanh{\alpha x} \sim \alpha x, \ \arctan{x} \sim x$. Так как функции слева и справа нечетны, помимо\
тривиального корня $\tilde{x}_0 = 0$ остальные (если существуют) связаны соотношением\
$\tilde{x}_i^+ = - \tilde{x}_i^-,\ \tilde{x}_i^+ > 0$. Отсюда и из графика можно сделать\
выводы, аналогичные выводам из конспекта семинара по задаче (1), причем интересует случай, когда
\salign[*]{\alpha = 1 + \varepsilon > 1,\ 0 < \varepsilon \ll 1,}
т. е. имеются две нетривиальные точки пересечения, и при малых $\varepsilon$ корни уравнения малы.\
\svg[0.7]{problem1.1}{График функций $\boldsymbol{\arctan{x}}$ и $\boldsymbol{\tanh{\alpha x}}$ при различных $\boldsymbol{\alpha}$}

Разложим обе части уравнения по степеням аргументов:
\salign[*]{\tanh{\alpha x} \approx (1 + \varepsilon)x - \frac13(1 + \varepsilon)^3x^3 + \frac2{15}(1 + \varepsilon)^5x^5, \quad  \arctan{x} \approx x - \frac13x^3 + \frac15x^5,}
\salign[*]{\varepsilon x - \frac13\left((1 + \varepsilon)^3 - 1\right)x^3 + \frac15\left(\frac23(1 + \varepsilon)^5 - 1\right)x^5 = 0.}
Из $\varepsilon \ll 1$ следует $(1 + \varepsilon)^n \approx 1 + n\varepsilon$ и, следовательно,
\salign[*]{\varepsilon x - \varepsilon x^3 + \left(\frac23\varepsilon - \frac1{15}\right)x^5 = 0.}
Разделим уравнение на $x$, домножим на $15$ для удобства, произведем замену $y = x^2$\
и решим приближенно полученное квадратное уравнение:
\salign[*]{y^2(10\varepsilon - 1) - 15\varepsilon y + 15\varepsilon = 0,}
\salign[*]{\tilde{y} = \frac{15\varepsilon \pm \sqrt{15^2\varepsilon^2 - 60\varepsilon(10\varepsilon - 1)}}{2(10\varepsilon - 1)} = \frac{\pm 30\sqrt{\varepsilon} \cdot \sqrt{1-\frac{39}4\varepsilon} + 15\varepsilon}{2(10\varepsilon - 1)} \approx \pm 15\sqrt{\varepsilon},}
а так как $y > 0$:
\salign[*]{\tilde{y} \approx 15\sqrt{\varepsilon}.}
Подставляя полученный результат в $\varepsilon = \alpha - 1$ и $x = \pm\sqrt{y}$, получаем ответ:
\salign[*]{\boxed{\tilde{x}_0 = 0,\ \tilde{x}_{1,2} \approx \pm\sqrt{15\sqrt{\alpha - 1}}}}
\subsubsection*{\Rnum{2}. $\alpha \gg 1$}
Найдем сначала положительный корень данного уравнения $\tilde{x}_1 > 0$. Так как $\alpha \gg 1$,
\salign{\tanh{\alpha \tilde{x}_1} = \frac{e^{\alpha \tilde{x}_1} - e^{-\alpha \tilde{x}_1}}{e^{\alpha \tilde{x}_1} + e^{-\alpha \tilde{x}_1}} = 1 - \frac{2}{e^{2\alpha \tilde{x}_1} + 1} = 1 - \varepsilon, \ 0 < \varepsilon < 1.}
В грубом приближении $\tanh{\alpha\tilde{x}_1} \approx 1, \ \tilde{x}_1 \approx \tan{1} \Rightarrow \varepsilon \ll 1$.
Выразим $\tilde{x}_1$ через $\varepsilon$:
\salign[*]{1 - \varepsilon = \arctan{\tilde{x}_1}, \quad \tilde{x}_1 = \tan{(1 - \varepsilon)} = \frac{\tan{1} - \tan{\varepsilon}}{1 + \tan{1}\tan{\varepsilon}}.}
Пусть $\tan{1} = \beta$. Так как $\beta\tan{\varepsilon} \ll 1$,
\salign{\tilde{x}_1 = \beta\frac{1 - \frac{\tan{\varepsilon}}{\beta}}{1 + \beta \tan{\varepsilon}} = \beta - \frac{(\beta^2 + 1)\tan{\varepsilon}}{1 + \beta\tan{\varepsilon}} \approx \beta - (\beta^2 + 1)\varepsilon.}
Теперь преобразуем (10) и подставим $\tilde{x}_1$ из (11):
\salign[*]{\varepsilon = \frac{2}{e^{2\alpha\tilde{x}_1 + 1}}, \quad \tilde{x}_1 = \frac{1}{2\alpha}\ln{\left(\frac{1}{\varepsilon} - 1\right)} \approx \beta - (\beta^2 + 1)\varepsilon.}
Применим метод итераций к полученному уравнению. Возьмем $\varepsilon_1 = 0$, тогда:
\salign[*]{\frac{1}{2\alpha}\ln{\left(\frac{1}{\varepsilon_{n+1}} - 1\right)} = \beta - (\beta^2 + 1)\varepsilon_n, \quad \varepsilon_2 = \frac{1}{e^{2\alpha\beta} + 1}, \quad \varepsilon_3 = \frac{1}{e^{2\alpha\beta}\left[e^{-(\beta^2 + 1)\frac{1}{e^{2\alpha\beta} + 1}}\right] + 1}.}
Уже при $\alpha = 5$
\salign[*]{\varepsilon_2 \approx 1.7 \cdot 10^{-7}, \ \frac{(\Delta\varepsilon)_2}{\varepsilon_2} = \frac{\varepsilon_3 - \varepsilon_2}{\varepsilon_2} \approx 10^{-5} \Rightarrow (\Delta\varepsilon)_2 \ll \varepsilon_2,}
а следовательно
\salign[*]{\varepsilon \approx \varepsilon_2 \approx \frac{1}{e^{2\alpha\beta} + 1}}
является достаточным приближением.

Подставляя полученную величину в (11), получаем приближенное выражение для $\tilde{x}_1$:
\salign[*]{\tilde{x}_1 \approx \beta - \frac{\beta^2 + 1}{e^{2\alpha\beta} + 1}.}
Так как в исходном уравнении правая и левая части --- нечетные функции, $\exists \tilde{x}_2 = -\tilde{x}_1$:
корень исходного уравнения, откуда, учитывая тривиальный корень $\tilde{x}_0 = 0$, ответ:
\salign[*]{\boxed{\tilde{x}_0 = 0,\ \tilde{x}_{1,2} \approx \pm\left(\beta - \frac{\beta^2 + 1}{e^{2\alpha\beta} + 1}\right),\ \beta = \tan{1}}}

\subsection*{Задача 2}
\svg[0.7]{problem2}{График функции $\boldsymbol{|\cos{x} + \alpha\sinc{x}|}$ при $\boldsymbol{\alpha = 3}$}
На \textbf{рис. 2} изображен график $|\cos{x} + \alpha\sinc{x}|$ при $\alpha = 3$ для наглядности.\
Из графика и/или тривиальных математических рассуждений видно, что на каждый интервал\
$\left(\pi k,\ \pi (k + 1)\right), k \in \mathbb{N} \cup \{0\}$ приходится зона:
\salign{K_{k} = \left(\pi k,\ \pi k + \varepsilon_{k}\right),\ \varepsilon_k > 0}
при этом оба слагаемых имеют одинаковый знак, и модуль можно раскрыть:
\salign[*]{\left|\cos{x} + \alpha \frac{\sin{x}}{x}\right| = |\cos{x}| + \alpha\frac{|\sin{x}|}{x}}
При $\alpha \ll 1$ и $k \gg 1$ (а, следовательно, и $x \gg 1$) ширина $(\Delta x)_k$ этих участков $(\Delta x)_k \ll 1$,\
а следовательно поведение $\sin{x}$ и $\cos{x}$ в этих зонах описывается аналогично поведению\
в окрестности нуля, т. е.:
\salign[*]{x \in K_{k},\ |\cos{x}| + \alpha\frac{|\sin{x}|}{x} \approx 1 - \frac{(x - \pi k)^2}{2} + \alpha \frac{(x-\pi k) - \frac{(x-\pi k)^3}{6}}{x},}
откуда получаем уравнение для правой границы зоны:
\salign[*]{x = \pi k + \varepsilon_k,\quad 1 - \frac{\varepsilon_k^2}{2} + \alpha \frac{\varepsilon_k - \frac{\varepsilon_k^3}{6}}{\pi k + \varepsilon_k} \approx 1,\quad -3\pi k\varepsilon_k^2 + \alpha \frac{6\varepsilon_k - \varepsilon_k^3}{1 + \frac{\varepsilon_k}{\pi k}} \approx 0.}
Раскладываем знаменатель по степеням $\varepsilon_k/\pi k$ :
\salign[*]{-3\pi k\varepsilon_k^2 + \alpha \left(6\varepsilon_k - \varepsilon_k^3\right)\left(1 - \frac{\varepsilon_k}{\pi k}\right) \approx 0 \approx -3\pi k\varepsilon_k^2 + 6\alpha\varepsilon_k - \frac{6\alpha\varepsilon_k^2}{\pi k},}
\salign[*]{-\left(\pi k + \frac{2\alpha}{\pi k}\right)\varepsilon_k + 2\alpha \approx 0,\quad \varepsilon_k \approx \frac{2\alpha}{\pi k + \frac{2\alpha}{\pi k}} \approx \frac{2\alpha}{\pi k}.}
Из (12) получаем
\salign[*]{(\Delta x)_k = (\pi k + \varepsilon_k) - \pi k = \varepsilon_k,}
ответ:
\salign[*]{\boxed{(\Delta x)_k = \frac{2\alpha}{\pi k}}}
\end{document}