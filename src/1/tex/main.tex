\documentclass[a4paper, 12pt]{article}

\def \srcdir{tex/}
\def \picdir{pic/}

\input{\srcdir properties}
\input{\srcdir macros}

\title{Домашняя работа \textnumero \input{\srcdir index}}
\author{\input{\srcdir author}}
\date{\input{\srcdir date}}

\begin{document}

\maketitle\thispagestyle{fancy}

\subsection*{Упражнение 1}
\subsubsection*{\Rnum{1}. $\alpha \gg 1$}
Пусть $\tilde{x}$ --- корень уравнения $x-1 = e^{-\alpha x}$, тогда,
\salign[*]{\forall x \hookrightarrow e^{-\alpha x} \geq 0 \Rightarrow \tilde{x} > 1 \Rightarrow e^{-\alpha \tilde{x}} < e^{-\alpha} \ll 1,}
следовательно, $0 < \tilde{x} - 1 \ll 1$, и $\tilde{x}$ можно представить в виде
\salign{\tilde{x} = 1 + \varepsilon, \quad 0 < \varepsilon \ll 1.}
От полученного данной подстановкой уравнения $\varepsilon = e^{-\alpha(1 + \varepsilon)}$ отбросим малый член:
\salign[*]{e^{-\alpha(1 + \varepsilon)} \approx e{-\alpha}, \quad -\alpha \approx \ln{\varepsilon},}
откуда получаем $\varepsilon \approx e^{-\alpha}$, и, подставляя $\varepsilon$ в (1), получаем ответ:
\salign[*]{\boxed{\tilde{x} \approx 1 + e^{-\alpha}}}

\subsubsection*{\Rnum{2}. $\alpha \ll 1$}
Пусть $\tilde{x}$ --- корень уравнения $x-1 = e^{-\alpha x}$, тогда,
\salign[*]{\forall x \hookrightarrow e^{-\alpha x} \geq 0 \Rightarrow \tilde{x} > 1 \Rightarrow |-\alpha \tilde{x}| \ll 1,}
следовательно, $0 < 1 - e^{-\alpha \tilde{x}} \ll 1$, и $e^{-\alpha \tilde{x}}$ можно представить в виде
\salign{e^{-\alpha \tilde{x}} = 1 - \varepsilon, \quad 0 < \varepsilon \ll 1,}
откуда $\tilde{x} = \frac{1}{\alpha}\ln{\frac{1}{1 - \varepsilon}}$, и, подстановкой (2) в исходное уравнение,
\salign{\tilde{x} = 2 - \varepsilon,}
\salign[*]{\alpha(2 - \varepsilon) = \ln{\frac{1}{1 - \varepsilon}}.}
Пренебрегая малой величиной, получаем
\salign[*]{(2 - \varepsilon)\alpha \approx 2\alpha, \quad 2\alpha \approx \ln{\frac{1}{1 - \varepsilon}}, \quad \varepsilon \approx 1 - e^{-2\alpha},}
откуда подстановкой в (3) получаем ответ:
\salign[*]{\boxed{\tilde{x} \approx 2 - e^{-2\alpha}}}

\subsection*{Упражнение 2}
\subsubsection*{\Rnum{1}. $\alpha \gg 1$}
Пусть $\tilde{x}$ --- корень уравнения $\ln{x} = e^{-\alpha x}$, тогда,
\salign[*]{\forall x \hookrightarrow e^{-\alpha x} \geq 0 \Rightarrow \ln{\tilde{x}} > 0 \Rightarrow \tilde{x} > 1 \Rightarrow e^{-\alpha \tilde{x}} < e^{-\alpha} \ll 1,}
следовательно, $0 < \ln{\tilde{x}} \ll 1$, и $\tilde{x}$ можно представить в виде
\salign{\tilde{x} = 1 + \varepsilon, \quad 0 < \varepsilon \ll 1.}
От полученного данной подстановкой уравнения $\ln{(1 + \varepsilon)} = e^{-\alpha(1 + \varepsilon)}$
возьмем экспоненту:\
\salign[*]{1 + \varepsilon = e^{e^{-\alpha(1 + \varepsilon)}},}
и, так как $\xi = e^{-\alpha(1+\varepsilon)} \ll 1$, разложим правую часть по степеням $\xi$:
\salign[*]{1 + \varepsilon = e^\xi \approx 1 + \xi + \frac{1}{2}\xi^2.}
Подставляя $\xi$ и пренебрегая малыми величинами, получаем
\salign[*]{\varepsilon \approx e^{-\alpha(1+\varepsilon)} + \frac{1}{2}e^{-2\alpha(1+\varepsilon)} \approx e^{-\alpha},}
и, подставляя $\varepsilon$ в (4), получаем ответ:
\salign[*]{\boxed{\tilde{x} \approx 1 + e^{-\alpha}}}

\subsubsection*{\Rnum{2}. $\alpha \ll 1$}
Пусть $\tilde{x}$ --- корень уравнения $x-1 = e^{-\alpha x}$, тогда,
\salign[*]{\forall x \hookrightarrow e^{-\alpha x} \geq 0 \Rightarrow \tilde{x} > 1 \Rightarrow |-\alpha \tilde{x}| \ll 1,}
следовательно, $0 < 1 - e^{-\alpha \tilde{x}} \ll 1$, и $e^{-\alpha \tilde{x}}$ можно представить в виде
\salign{e^{-\alpha \tilde{x}} = 1 - \varepsilon, \quad 0 < \varepsilon \ll 1,}
откуда $\tilde{x} = \frac{1}{\alpha}\ln{\frac{1}{1 - \varepsilon}}$, и, подстановкой (5) в исходное уравнение,
\salign{\tilde{x} = e^{1 - \varepsilon},}
\salign[*]{\alpha e^{1 - \varepsilon} = \ln{\frac{1}{1 - \varepsilon}}.}
Пренебрегая малой величиной, получаем
\salign[*]{\alpha e^{1-\varepsilon} \approx \alpha e, \quad \alpha e \approx \ln{\frac{1}{1 - \varepsilon}}, \quad \varepsilon = 1 - e^{-e\alpha},}
откуда подстановкой в (6) получаем ответ:
\salign[*]{\boxed{\tilde{x} \approx e^{e^{-e\alpha}}}}

\subsection*{Упражнение 3}
\subsubsection*{\Rnum{1}. $\tilde{x}_1(\lambda)$}
Пусть $0 > \tilde{x} > -1$ --- корень уравнения $xe^x = \lambda$. При малых $x$ таких, что $|x| \ll 1$,\
\salign[*]{x e^x \approx x(1 + x + \frac{1}{2}x^2) \approx x,}
следовательно, $\tilde{x} \approx \tilde{x}e^x \approx \lambda$, ответ:
\salign[*]{\boxed{\tilde{x} \approx \lambda}}
\subsubsection*{\Rnum{2}. $\tilde{x}_2(\lambda)$}
Домножим на $-1$ и прологарифмируем обе части уравнения:
\salign[*]{x = \ln{(-\lambda)} - \ln{(-x)}.}
Чтобы записи были приятнее глазу, перепишем уравнение в следующем виде:
\salign{y = -x, \quad \xi = -\ln{(-\lambda)}, \quad y = \xi + \ln{y}.}

Пусть $\tilde{y}$ --- корень данного уравнения, удовлетворяющий условию задачи\
$x_2(\lambda) < -1 \Leftrightarrow \tilde{y} > 1$. Применим метод итераций к (7) и докажем, что\
последовательность $\{y_n\}$ сходится в $\tilde{y}$.

Пусть $y_1 = 1$, тогда
\salign[*]{y_2 = \xi + \ln{y_1}, \quad (\Delta y)_1 = y_2 - y_1 = \xi - 1.}
Так как $|\lambda| \ll 1$, $\xi > 1$ и, следовательно,
\salign[*]{y_2 > y_1, \quad (\Delta y)_1 > 0.}
Докажем ограниченность сверху для $\{y_n\}$. Предположим, что $\exists n: y_n \geq \tilde{y}$, тогда
\salign[*]{y_n = \xi + \ln{y_{n-1}}, \quad y_{n-1} = e^{y_n - \xi} = e^{(\tilde{y} - \xi) + (y_n - \tilde{y})} = \tilde{y}e^{y_n - \tilde{y}} \geq \tilde{y},}
откуда по индукции:
\salign[*]{\exists n: y_n \geq \tilde{y} \Rightarrow \forall k \in \mathbb{N}, \ k \leq n \hookrightarrow y_k \geq \tilde{y},}
что противоречит начальным условиям $\tilde{y} > 1, \ y_1 = 1$, следовательно $\{y_n\}$\
ограниченна сверху, и
\salign{\forall n \in \mathbb{N} \hookrightarrow y_n < \tilde{y}.}

Докажем, что последовательность $\{y_n\}$ монотонно возрастает:
\salign[*]{(\Delta y)_n = y_{n+1} - y_n = (\xi + \ln{y_n}) - (\xi + \ln{y_{n-1}}) = \ln{\frac{y_n}{y_{n-1}}} = \ln{\left(1 + \frac{(\Delta y)_{n-1}}{y_{n-1}}\right)},}
\salign[*]{(\Delta y)_{n-1} > 0 \Rightarrow \ln\left(1 + \frac{(\Delta y)_{n-1}}{y_{n-1}}\right) > 0 \Rightarrow (\Delta y)_n > 0,}
откуда по индукции:
\salign[*]{(\Delta y)_1 > 0 \Rightarrow \forall n \in \mathbb{N} \hookrightarrow (\Delta y)_n > 0 \Leftrightarrow \forall n \in \mathbb{N} \hookrightarrow y_{n+1} > y_n,}
т.е. $\{y_n\}$ монотонно возрастает.

Oтсюда и из условия (8) ограниченности последовательности $\{y_n\}$
\salign[*]{\exists \lim_{n \to \infty} y_n = \hat{y}, \quad \hat{y} \leq \tilde{y}.}
Переходя к пределу в формуле $y_{n+1} = \xi + \ln{y_n}$ при $n \to \infty$, получаем\
$[*]\hat{y} = \xi + \ln{\hat{y}}$, т.е. $\hat{y} = \tilde{y}$. $\blacksquare$

То, с какой точностью записывать ответ, сильно зависит от $\lambda$. Например, при\
$x_1 = -1$ для $\lambda = -e^{-4}$, с точностью до трёх значащих цифр
\salign[*]{\tilde{x}_2(-e^4) \approx x_6 = -(\xi + \ln{(\xi + \ln{(\xi + \ln{(\xi + \ln{\xi})})})}), \ \xi = -\ln{(-\lambda)} = 4,}
a для $\lambda = -e^{-8}$ ту же точность получаем при
\salign[*]{\tilde{x}_2(-e^8) \approx x_4 = -(\xi + \ln{(\xi + \ln{\xi})}), \ \xi = -\ln{(-\lambda)} = 8.}
\textbf{Буду считать, что $\lambda = -e^{-4} \approx -0.02$ удовлетворяет условию $|\lambda| \ll 1$ и запишу в ответ $\tilde{x}_2(\lambda) \approx x_6$:}
\salign[*]{\boxed{\tilde{x}_2(\lambda) \approx -(\xi + \ln{(\xi + \ln{(\xi + \ln{(\xi + \ln{\xi})})})}), \ \xi = -\ln{(-\lambda)}}}
\subsection*{Упражнение 4}
Пусть $\tilde{x}$ --- корень данного уравнения, а $\lambda = - \frac{1}{e} + \delta, 0 < \delta \ll 1$,\
тогда $\tilde{x}$ можно представить в виде
\salign{\tilde{x} = -1 + \varepsilon, \quad |\varepsilon| \ll 1.}
Домножим обе части уравнения на $e$ и разложим левую часть по степеням $\varepsilon$:
\salign[*]{(-1 + \varepsilon)e^\varepsilon \approx (-1 + \varepsilon)(1 + \varepsilon + \frac{1}{2}\varepsilon^2) \approx -1 + \delta e,}
\salign[]{\frac{1}{2}\varepsilon^2 = \delta e, \quad \varepsilon = \pm \sqrt{2\delta e} = \pm \sqrt{2(\lambda e + 1)},}
отсюда, подставляя в (9) и учитывая, что $\tilde{x}_1(\lambda) > \tilde{x}_2(\lambda)$, ответ:
\salign[*]{\boxed{\tilde{x}_1(\lambda) = -1 + \sqrt{2(\lambda e + 1)}, \quad \tilde{x}_2(\lambda) = -1 - \sqrt{2(\lambda e + 1)}}}
\lipsum[7-8]

\subsection*{Задача 1}
\lipsum[9-10]

\subsection*{Задача 2}
\lipsum[11-12]



\end{document}