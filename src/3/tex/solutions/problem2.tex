\def \picdir{pic/}
\subsection*{Задача 2}
Преобразуем интеграл:
\feq{I(\lambda,\epsilon,s) = \int_0^{+\infty} x^se^{-\epsilon x}e^{- \lambda(1 - \cos{x})}\; \df x = \int_0^{+\infty} x^se^{-\epsilon x - \lambda(1 - \cos{x})}\; \df x.}
Пусть $f(x)$ --- степень при экспоненте, $\cg(x)$ --- подынтегральная функция:
\feq{f(x) = -\epsilon x - \lambda(1 - \cos{x}),\quad \cg(x) = e^{f(x)}.}
Представим $f(x)$ в виде
\feq{f(x) = -\lambda \tilde{f}(x),\quad \tilde{f}(x) = \frac{\epsilon}{\lambda}x + (1 - \cos{x}).}
Найдем первую и вторую производные $\tilde{f}(x)$:
\begin{equation*}
  \tilde{f}'(x) = \frac{\epsilon}{\lambda} + \sin{x},\quad \tilde{f}''(x) = \cos{x}.
\end{equation*}
Первая производная обнуляется при $\sin{x} = \frac{\epsilon}{\lambda}$, а следовательно, для\
точек
\begin{equation*}
  \hat{x}_1(k) \approx (2k - 1)\pi + \frac{\epsilon}{\lambda},\quad \hat{x}_2(k) \approx 2 \pi k - \frac{\epsilon}{\lambda},\ k \in \mathbb{N}
\end{equation*}
выполняется
\begin{equation*}
  \tilde{f}'(\hat{x}_{1,2}) = 0,\quad \tilde{f}''(\hat{x}_1) \approx - 1 + \frac{\epsilon^2}{2\lambda^2} \ne 0,\quad \tilde{f}''(\hat{x}_2) \approx 1 - \frac{\epsilon^2}{2 \lambda^2},
\end{equation*}
а так как
\begin{gather*}
  \tilde{f}(\hat{x}_1(k)) \approx \frac{\epsilon}{\lambda}(2k-1)\pi + \frac{\epsilon^2}2{\lambda^2} + 2,\quad \tilde{f}(\hat{x}_2(k)) \approx \frac{\epsilon}{\lambda}2 \pi k - \frac{\epsilon^2}{2\lambda^2},\\
  \tilde{f}(\hat{x}_1(k)) - \tilde{f}(\hat{x}_2(k)) = 2 + \frac{\epsilon^2}{\lambda^2} - \frac{\pi \epsilon}{\lambda} \approx 2 \Rightarrow g(\hat{x}_1(k)) \ll g(\hat{x}_2(k)),
\end{gather*}
т.е в точках $\hat{x}_2(k)$ подынтегральная функция $g(x)$ имеет максимумы. Так как $\lambda \gg 1$,\
максимумы резкие, и для каждого можно применить метод перевала:
\begin{align*}
  i_k = \left(\hat{x}_2(k) \right)^se^{-\lambda \tilde{f}(\hat{x}_2(k))}\sqrt{\frac{2 \pi}{\lambda| \tilde{f}''(\hat{x}_2(k))|}} &\approx \left(2 \pi k - \frac{\epsilon}{\lambda}\right)^se^{-2 \pi \epsilon k + \frac{\epsilon^2}{2 \lambda}} \sqrt{\frac{2 \pi}{\lambda - \frac{\epsilon^2}{2 \lambda}}}\\
  &\approx (2 \pi k)^se^{-2 \pi \epsilon k}\sqrt{\frac{2 \pi}{\lambda}},
\end{align*}
\begin{equation*}
I(\lambda,\epsilon,s) \approx \sum_{k=1}^{\infty} i_k \approx \sum_{k=1}^{\infty} (2 \pi k)^se^{-2 \pi \epsilon k}\sqrt{\frac{2 \pi}{\lambda}} = (2 \pi)^s\sqrt{\frac{2 \pi}{\lambda}} \sum_{k=1}^{\infty} k^se^{-2 \pi \epsilon k}.
\end{equation*}
Полученная сумма была посчитана в \textbf{Упр. 4 домашней работы \textnumero2}:
\begin{equation*}
  \sum_{k=1}^{\infty} k^se^{-2 \pi \epsilon k} \approx \frac{1}{(2 \pi \epsilon)^{s+1}}\Gamma(s + 1),\quad I(\lambda,\epsilon,s) \approx \frac{1}{\epsilon^{s+1}}\sqrt{\frac{1}{2 \pi\lambda}}\Gamma(s+1),
\end{equation*}
ответ:
\begin{equation*}
\boxed{I(\lambda,\epsilon,s) \approx \frac{1}{\epsilon^{s+1}}\sqrt{\frac{1}{2 \pi\lambda}}\Gamma(s+1)}
\end{equation*}